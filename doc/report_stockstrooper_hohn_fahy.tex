%%%%%%%%%%%%%%%%%%%%%%%%%%%%%%%%%%%%%%%%%%%%%%%%%%%%%%%%%%%%%%%%%%%%%
%
% Axel Fahy & Höhn Rudolf
% Automn Semester MSE
% IV
%
%%%%%%%%%%%%%%%%%%%%%%%%%%%%%%%%%%%%%%%%%%%%%%%%%%%%%%%%%%%%%%%%%%%%%%

\documentclass[11pt]{report}
%\documentclass[twoside, openright]{report}                           % To print on twoside
\usepackage[a4paper]{geometry}
\usepackage[T1]{fontenc}
\usepackage[utf8]{inputenc}
\usepackage[myheadings]{fullpage}
\usepackage{lastpage}
\usepackage{graphicx, wrapfig, subcaption, setspace, booktabs}
\usepackage[font=small, labelfont=bf]{caption}
\usepackage{fourier}
\usepackage[protrusion=true, expansion=true]{microtype}
\usepackage{amsmath}
\usepackage{sectsty}
\usepackage{hyperref}                                                 % Links are clickable
\usepackage{url, lipsum}
\usepackage{parskip}                                                  % Remove identation
\usepackage{emptypage}                                                % Do not print page number on empty pages
\usepackage{float}
\usepackage[absolute]{textpos}
\usepackage{fancyvrb}

\bibliographystyle{unsrt}

%-------------------------------------------------------------------------------
% MARGIN SETTINGS
%-------------------------------------------------------------------------------
\geometry{
	paper=a4paper, % Change to letterpaper for US letter
	inner=2.5cm, % Inner margin
	outer=3.5cm, % Outer margin
	bindingoffset=1.5cm, % Binding offset
	%top=1.5cm, % Top margin
	%bottom=1.5cm, % Bottom margin
	%showframe,% show how the type block is set on the page
}

%-------------------------------------------------------------------------------
% LIST DIRECTORIES
%-------------------------------------------------------------------------------
\usepackage[dvipsnames]{xcolor}                                       % Color names
\usepackage{dirtree}                                                  % List directories
\renewcommand*\DTstylecomment{\rmfamily\color{Gray}\textsc}           % Color of comments
\renewcommand*\DTstyle{\ttfamily\textcolor{OliveGreen}}               % Color of stucture
\setlength{\DTbaselineskip}{12pt}                                     % Skip between lines
\DTsetlength{1em}{1em}{0.1em}{1pt}{2pt}                               % Lines format

%-------------------------------------------------------------------------------
% CODE FORMAT
%-------------------------------------------------------------------------------
\usepackage{listings}                                                 % Code listings
\usepackage{listingsutf8}
\lstset{inputencoding=utf8/latin1}
\usepackage[dvipsnames]{xcolor}
\definecolor{mygreen}{rgb}{0,0.6,0}
\definecolor{mygray}{rgb}{0.5,0.5,0.5}
\definecolor{mymauve}{rgb}{0.58,0,0.82}
\definecolor{bggray}{rgb}{0.95, 0.95, 0.95}
\lstset{ %
    backgroundcolor=\color{bggray},   % choose the background color; you must add \usepackage{color} or \usepackage{xcolor}
    basicstyle=\footnotesize,        % the size of the fonts that are used for the code
    breakatwhitespace=false,         % sets if automatic breaks should only happen at whitespace
    breaklines=true,                 % sets automatic line breaking
    captionpos=b,                    % sets the caption-position to bottom
    commentstyle=\color{mygreen},    % comment style
    deletekeywords={}            % if you want to delete keywords from the given language
    escapeinside={\%*}{*},          % if you want to add LaTeX within your code
    extendedchars=true,              % lets you use non-ASCII characters; for 8-bits encodings only, does not work with UTF-8
    frame=single,                    % adds a frame around the code
    frameround=tttt                  % tttt for having the corner round.
    keepspaces=true,                 % keeps spaces in text, useful for keeping indentation of code (possibly needs columns=flexible)
    keywordstyle=\color{blue},       % keyword style
    language=html,                 % the language of the code
    morekeywords={*},            % if you want to add more keywords to the set
    numbers=none,                    % where to put the line-numbers; possible values are (none, left, right)
    numbersep=5pt,                   % how far the line-numbers are from the code
    numberstyle=\tiny\color{mygray}, % the style that is used for the line-numbers
    rulecolor=\color{black},         % if not set, the frame-color may be changed on line-breaks within not-black text (e.g. comments (green here))
    showspaces=false,                % show spaces everywhere adding particular underscores; it overrides 'showstringspaces'
    showstringspaces=false,          % underline spaces within strings only
    showtabs=false,                  % show tabs within strings adding particular underscores
    stepnumber=1,                    % the step between two line-numbers. If it's 1, each line will be numbered
    stringstyle=\color{mymauve},     % string literal style
    tabsize=2,                       % sets default tabsize to 2 spaces
    title=\lstname}                 % show the filename of files included with \lstinputlisting; also try caption instead of title

%-------------------------------------------------------------------------------
% COMMAND LINE FORMAT
%-------------------------------------------------------------------------------
\lstdefinestyle{CommandLineStyle}{
    backgroundcolor=\color{black},
    basicstyle=\color{white}\footnotesize\ttfamily,
    breakatwhitespace=false,
    breaklines=true,
    captionpos=b,
    deletekeywords={},
    escapeinside={\%*}{*},
    extendedchars=true,
    frame=none,
    keepspaces=true,
    numbers=none,
    rulecolor=\color{black},
    showspaces=false,
    showstringspaces=false,
    showtabs=false,
    tabsize=2,
    deletekeywords={default}}
%-------------------------------------------------------------------------------
% INLINE FORMAT
%-------------------------------------------------------------------------------
\lstdefinestyle{InlineStyle}{
    basicstyle=\footnotesize,
    breakatwhitespace=false,
    breaklines=true,
    captionpos=b,
    deletekeywords={},
    escapeinside={\%*}{*},
    extendedchars=true,
    frame=none,
    keepspaces=true,
    keywordstyle=\color{blue},
    language=java,
    morekeywords={*},
    showspaces=false,
    showstringspaces=false,
    showtabs=false,
    tabsize=2}


\newcommand{\HRule}[1]{\rule{\linewidth}{#1}}
\onehalfspacing{}
\setcounter{tocdepth}{5}
\setcounter{secnumdepth}{5}

\usepackage{titlesec}
%\titleformat{\chapter}{}{}{0em}{\bf\LARGE}      % Remove the 'Chapter' before each chapter
%\titleformat{\chapter}{\normalfont\huge}{\thechapter.}{20pt}{\huge\it}
\titleformat{\chapter}{\normalfont\huge}{\thechapter.}{20pt}{\huge}
\titlespacing*{\chapter}{0pt}{-50pt}{30pt}      % Change 'before' spacing (default is 50pt) and 'after' spacing (default is 40pt)
%\makeatletter
%\renewcommand{\@makechapterhead}[1]{%
%\vspace*{0 pt}%
%\bfseries\Huge\thechapter.\ #1
%\par\nobreak\vspace{40 pt}}}
%\makeatother

\usepackage{pdfpages}                                                 % To include another pdf

%-------------------------------------------------------------------------------
% HEADER & FOOTER
%-------------------------------------------------------------------------------
\usepackage{fancyhdr}
\pagestyle{fancy}
\fancyhf{}
\setlength\headheight{15pt}
\fancyhead[L]{\textsl{\leftmark}}
\fancyfoot[L]{\textsl{Axel} FAHY \& \textsl{Rudolf} HÖHN}
\fancyfoot[R]{\thepage}
\renewcommand{\footrulewidth}{0.4pt}    % Horizontale line for footer
% Redefine the plain page style
\fancypagestyle{plain}{
  \fancyhf{}
  \fancyfoot[R]{\thepage}
  \renewcommand{\headrulewidth}{0pt}    % Line at the header invisible
  \renewcommand{\footrulewidth}{0pt}    % Line at the footer visible
}

\begin{document}
%-------------------------------------------------------------------------------
% TITLE PAGE
%-------------------------------------------------------------------------------
\title{Stocks Trooper\\Information Visualization\\MSE}
\date{\today}
\author{Axel Fahy \& Rudolf Höhn}
\maketitle

\tableofcontents

\pagestyle{fancy}     % Print page number again

%-------------------------------------------------------------------------------
% CHAPTER INTRODUCTION
%-------------------------------------------------------------------------------
\chapter{Introduction}
\label{chapter:introduction}
In this chapter, we define what is the aim of the project, the different features and mockups we are implementing and finally, the different technologies we are using.

%-------------------------------------------------------------------------------
%-------------------------------------------------------------------------------
\section{Aim of the project}
This projects aims to detect important events on stocks values and present them on a GUI.\@ An event is understood as a ''\textit{significant decrease or increase in a short amount of time}''. For each event, the application will suggest relevant articles to help the user have a better understanding.

%-------------------------------------------------------------------------------
%-------------------------------------------------------------------------------
\section{Features to implement}
These following features must appear in the application:
\begin{enumerate}
    \item Display the stocks values of a particular index between two dates
    \item Display on a timeline the events detected in the stocks values
    \item Display some news articles related to the event
\end{enumerate}

%-------------------------------------------------------------------------------
%-------------------------------------------------------------------------------
\section{Technologies used}
Python and AngularJS are the two main programming languagues we are using in this application. In addition to the language, we are using the following libraries:
\begin{itemize}
    \item Pandas
    \item XXX (TODO complete here)
\end{itemize}


%-------------------------------------------------------------------------------
% CHAPTER SPECIFICATIONS
%-------------------------------------------------------------------------------
\chapter{Specifications}
\label{chapter:specifications}
In this chapter, we describe precisely the IT related specifications (e.g. REST routes, data format, GUI).

%-------------------------------------------------------------------------------
%-------------------------------------------------------------------------------
\section{System architecture}
TODO graphical representation of the system

%-------------------------------------------------------------------------------
%-------------------------------------------------------------------------------
\section{REST API}

%-------------------------------------------------------------------------------
\subsection{Routes and format}

\subsubsection*{GET events}
This route sends all events within a period for one index. Default is one year from now.
\begin{verbatim}
/events/:index[/:datestart/:dateend]

[
    {"date": "20160111"},
    {"date": "20160212"},
    {"date": "20160313"}
]
\end{verbatim}
\subsubsection*{GET stocks}
This route sends all the stocks data for one index (market index).
\begin{verbatim}
/stocks/:index

[
    {
        "date": "20160111"
        "value": 14.04
    },
    {
        "date": "20160112"
        "value": 15.04
    },
        "date": "20160113"
        "value": 19.04
    }
]
\end{verbatim}

%-------------------------------------------------------------------------------
\subsection{CORS}


%-------------------------------------------------------------------------------
% CHAPTER CONCLUSION
%-------------------------------------------------------------------------------
\chapter{Conclusion}
\label{chapter:conclusions}

%-------------------------------------------------------------------------------
%-------------------------------------------------------------------------------
\section{Project status}

%-------------------------------------------------------------------------------
%-------------------------------------------------------------------------------
\section{System limitations}

%-------------------------------------------------------------------------------
%-------------------------------------------------------------------------------
\section{Future work}


\appendix
%-------------------------------------------------------------------------------
% APPENDIX 1
%-------------------------------------------------------------------------------
\include{Appendices/Appendix1}

% \begin{table}
% \caption{Example of a Machine Learning algorithm prediction abilities}
% \label{tab:ml-example}
% \centering
% \begin{tabular}{l l l | l}
% \toprule
% \tabhead{Color} & \tabhead{Diameter $[cm]$} & \tabhead{Fruit} & \tabhead{Chances of good prediction}\\
% \midrule
% Red & 5 & Apple & Good\\
% Orange & 5 & Orange & Good\\
% Orange & 8 & Apple & Very bad\\
% Yellow & 10 & Orange & Bad\\
% \bottomrule\\
% \end{tabular}
% \end{table}

\end{document}
