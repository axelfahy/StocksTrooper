\chapter{Conclusion}
\label{chapter:conclusions}

%-------------------------------------------------------------------------------
%-------------------------------------------------------------------------------
\section{Project status}

%-------------------------------------------------------------------------------
%-------------------------------------------------------------------------------
\section{System limitations}

%-------------------------------------------------------------------------------
%-------------------------------------------------------------------------------
\section{Future work}

\subsection{Improvement on the UX}
Here is a non-exaustive list of things that can be improved within the UX aspect.
\begin{itemize}
    \item \textbf{Override the defaults}: the default range of 1 year, the number of events, etc. Maybe we can integrate these customizations and preferences as a user profile.
    \item \textbf{News time range around the event}: on an event time range of 1 year, the list of news is filled with articles that might have been published $\pm$ 2 months around the event. This is a ratio of about $\frac{1}{6}$ of the event time range to retrieve the news. It might be relevant to remove this 2 months security, but instead to keep this ratio and always provide a $\frac{1}{6}$ ratio between the events and the news time range. For example, for an event time of 1 month, we could have articles of $\pm$ 5 days around the event.
    \item \textbf{History of use}: it might be interesting to have some sort of a list that contains all the requests made by the user. This way, he can go back to a previous state of the application and analyze the data as previously.
\end{itemize}

Although this list does not contain all the possible improvements on the UX, the best way to find other aspects to take into consideration is to do a real user testing with real end-users.

\subsection{Diversify the sources of the data}
